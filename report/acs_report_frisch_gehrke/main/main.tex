\documentclass[
pdfa=false,  % to avoid compilation error
color=9c,
logo=body,
class=article,
marginpar=false,
logofile=../logos/tuda_logo.pdf
]{tudapub}

% ---------- Packages ---------- % 

\usepackage[main=english, ngerman]{babel}
\usepackage{lipsum} % for dummy text
\usepackage{subfiles} % Best loaded last in the preamble
\usepackage{caption} % To center captions
\usepackage{tikz} % For drawing
\usepackage{float} % To fix figure positions where they should be...
\usetikzlibrary{external} % To externalize tikz figures
\usetikzlibrary{arrows} % For fancy errors in tikz figures
\usepackage{amsmath} % Use text in math mode
\usepackage[toc,page]{appendix} % For appendix
\usepackage[round]{natbib} % For citation style

% ---------- Notes ---------- % 

% TODO: Do we want two columns?


% ---------- Make Title ---------- % 

\author{Yannik P. Frisch, Maximilian A. Gehrke}
\title{Predicting Task Performance - How to Quantify Metacognition Using Drawn Confidence Distributions}
\date{March 10, 2020}
\subtitle{Applied Cognitive Science}
\addTitleBoxLogo*{\includegraphics[width=.7\linewidth]{../logos/ccs_logo.png}}


\begin{document}
\maketitle

% ---------- Abstract ---------- % 
\begin{abstract}
	Several studies suggest that humans are rather poor when judging their own performance after they have fulfilled a task. We present a new method to use drawn probability density functions to measure the subjects' confidence of their performance on several tasks of a questionnaire that we designed especially for that purpose. Subsequently we rate their answers using a scoring function that is unknown to the subjects, before we evaluate their confidence by statistical measurements of the discrepancy between their expected and their actual performance on the tasks. After evaluating these, we come to the conclusion that [... TODO] and give suggestions for further experiments on this topic, using a wider variety of tasks and a higher number of subjects.
\end{abstract}

\section{Introduction (Yannik P. Frisch)}
	\label{sec:introduction}
	\subfile{../sections/introduction.tex}

\newpage
\section{Method (Maximilian A. Gehrke)}
	\label{sec:method}
	\subfile{../sections/method.tex}

\newpage
\section{Results (Yannik P. Frisch)}
	\label{sec:results}
	\subfile{../sections/results.tex}

\newpage
\section{Discussion (Maximilian A. Gehrke)}
	\label{sec:discussion}
	\subfile{../sections/discussion.tex}


% ---------- Bibliography ---------- %
\newpage
\bibliography{../bib/references.bib}{}
\bibliographystyle{plainnat}


% ---------- Appendix ---------- %
\newpage
\begin{appendices}
	\section{The Questionnaire}
		\label{appendix:questionnaire}
		\subfile{../sections/questionnaire.tex}
	\newpage
	\section{Individual Results}
		\label{appendix:individual_results}
		\subfile{../sections/individual_results.tex}
\end{appendices}


\end{document}